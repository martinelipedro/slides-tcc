\documentclass[aspectratio=169]{beamer}
\usepackage[utf8]{inputenc}

%-------------------------Packages------------------------
\usepackage[utf8]{inputenc}
\usepackage{subcaption}
\usepackage{amsmath}
%---------------------------------------------------------
% -----------Define theme and color scheme----------------

\usetheme[sidebarleft]{IIST} 
\usepackage{xcolor}

\setbeamercolor{title}{fg=black}
\setbeamertemplate{footline}{}
\logo{\includegraphics[height=1.5cm]{figures/logo.png}}


% ---------Information on the title page------------------
\title[$k$-limitadas] {Contagem de funções $k$-limitadas em $[n]$}
\author[] {Tiago Trindade \and Pedro Martineli}
\institute[IAL] {Instituto Alpha Lumen}
\date[Novembro, 2024]{26 de novembro de 2024}
%------------------------------------------------------------

%------------------------------------------------------------
%The next block of commands puts the table of contents at the beginning of each section and highlights the current section:

% \AtBeginSection[]
% {
%   \begin{frame}
%     \frametitle{Table of Contents}
%     \tableofcontents[currentsection]
%   \end{frame}
% }
%------------------------------------------------------------

\begin{document}

\frame{\titlepage}  % Creates title page

%--------table of contents after title page---------------
\begin{frame}
\frametitle{Conteúdo}
\tableofcontents
\end{frame}
%---------------------------------------------------------

\section{Introdução}

\begin{frame}
\frametitle{Funções}
\end{frame}

\section{Funções $(\alpha, \lambda)$-limitadas}
\section{Resultados nas \textit{matrizes de toeplitz}}

\section{Working with Texts}

%---------------------------------------------------------
%Changing visivility of the text
\begin{frame}
\frametitle{Slide 1}
Some itemized text...

\begin{itemize}
    \item<1-> First line comes first. 
    \item<2-> Second line joins the first in next slide.
    \item<3-> Third line joins the rest in next slide.
\end{itemize}

\end{frame}

%---------------------------------------------------------

%---------------------------------------------------------
%Highlighting text
\begin{frame}
  \frametitle{Types of Blocks}
  
  This is a brief introduction of \alert{Blocks}.
  
  \begin{block}{Definition}
  A simple definition block.
  \end{block}
  
  \begin{alertblock}{Alert}
  A simple alert block.
  \end{alertblock}
  
  \begin{examples}
  A simple example block.
  \end{examples}
\end{frame}
%---------------------------------------------------------



\section{Two Columns and Equations}
%---------------------------------------------------------
%Two columns
\begin{frame}
\frametitle{Two Columns with Footnote and Image}

\begin{columns}

\column{0.5\textwidth}
Indian Institute of Space Science and Technology (IIST), situated at Thiruvananthapuram is a Deemed to be University under Section 3 of the UGC Act 1956. IIST functions as an autonomous body under the Department of Space, Government of India \footnotemark[1]. 

\column{0.5\textwidth}
\begin{figure}
  \includegraphics[width=0.8\textwidth]{figures/Indian_Institute_of_Space_Science_and_Technology_Logo.png}
  \caption{IIST Logo}
  \label{fig:iistlogo}
\end{figure}
\end{columns}
\footnotetext[1]{https://www.iist.ac.in/aboutus/institute}
\end{frame}
%---------------------------------------------------------

%---------------------------------------------------------
%Equations
\begin{frame}
  \frametitle{Frame with a sample equation}
  This slide is to test mathematical formulas \pause
  \begin{equation}
    \begin{gathered}
        \nabla \cdot \vec{u}\, = \,0,\\
        \rho\left(\frac{\partial\vec{u}}{\partial t} \,+\,\vec{u} \cdot \nabla \vec{u}\right)\, =\, -\, \nabla P\, + \,\mu \,\nabla^2 \vec{u}\, +\, \rho\vec{f}
    \end{gathered}
  \end{equation}
\end{frame}
%---------------------------------------------------------


\section{Subfigures and Tables}
%---------------------------------------------------------
%Subfigures
\begin{frame}
  \frametitle{Use of Subfigures}
  \begin{figure}
    \centering
    \begin{subfigure}[b]{0.35\textwidth}
      \centering
      \includegraphics[width=\textwidth]{figures/Indian_Institute_of_Space_Science_and_Technology_Logo.png}
      \caption{1st Image}
      \label{fig:img1}
    \end{subfigure}
    \hspace{0.3cm}
    \begin{subfigure}[b]{0.35\textwidth}
      \centering
      \includegraphics[width=\textwidth]{figures/Indian_Institute_of_Space_Science_and_Technology_Logo.png}
      \caption{2nd Image}
      \label{fig:img2}
    \end{subfigure}
  \end{figure}
\end{frame}
%---------------------------------------------------------

%---------------------------------------------------------
%Tables
\begin{frame}
  \frametitle{Use of Tables}
  \begin{table}
    \centering
    \begin{tabular}{||c|c|c||}
      \hline
      \textbf{Heading 1} & \textbf{Heading 2} & \textbf{Heading 3}\\
      \hline
      Text & Text & Text \\
      \hline
      Text & Text & Text \\
      \hline
      Text & Text & Text \\
      \hline
    \end{tabular}
    \caption{Sample Table}

  \end{table}
\end{frame}
%---------------------------------------------------------
\end{document}